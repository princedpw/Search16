\section*{Collaboration Plan}
\label{sec:timeline}

The two PIs are uniquely qualified to carry out the proposed research.
Both PIs are experts in type systems and logic.  PI Zdancewic has been
developing a pure type-and-example-based system that will form the
initial kernel of the proposed work.  PI Walker brings experience in
data-driven synthesis from his past NSF-sponsored work on the PADS
project.  Together PIs Walker and Zdancewic have already begun
to work together and are currently making progress on the technical
components of this proposal.
%~\cite{fisher+:jacm,fisher+:popl06,mandelbaum+:pads-ml,fisher+:dirttoshovels,padsweb,daly+:pads-demo}.

% The PIs have worked together successfully in the past and the
% collaboration will get off to a quick start with PI Walker will be on
% sabbatical at Pi Zdancewic's institution (University of Pennsylvania)
% in the 2015-2016 academic year.

During the academic year of 2015-2016, PI Walker is on sabbatical at
the University of Pennsylvania.  There he and PI Zdancewic have
already begun a collaboration on this project.  They will work closely
together for the remainder of the year.  They expect this concentrated
time together will accelerate the research productivity of the PIs and
their students.  When the sabbatical is over, the PIs will set up
weekly Skype video conference calls together and with their students
to discuss research on program synthesis.  These meetings will
continue throughout the duration of the grant.  The PIs will also
schedule periodic in-person meetings.  (The University of Pennsylvania
and Princeton University are only 1 hour away by car or train.)  We
have budgeted travel funds both for going to conferences and to
facilitate commutes back and forth between institutions when
necessary.

\paragraph*{Software-development Infrastructure.}
We have set up a joint git 
repository, which holds our code.  Git will store a main software
development branch and will allow us to generate experimental branches,
which may be later merged back in to the core.  Our git repository also
includes test suites, notes, technical reports, and documents relevant 
to the project (such as this proposal).

\paragraph*{Research Tasks and Timeline.}
Table~\ref{fig:timeline} (see the following page)
summarizes the major pieces of the research
in terms of area.   The table lists the central research tasks and
provides expected start and finish times. 

\begin{table}[h]
\centering
\begin{tabular}{|l||c|c|}
\hline
Research Task & Start & Finish\\
\hline
\multicolumn{3}{|c|}{Core Type-theoretic Synthesis Algorithms}\\
\hline
Negative examples & In progress & Year 1 \\
Performance optimization & In progress & Year 4 \\
Example minimization & In progress & Year 4 \\
\hline
\multicolumn{3}{|c|}{Advanced Type Systems}\\
\hline
Polymorphism, GADTs, refinements & In Progress & Year 1 \\
Regular expression typing, Dependency & Year 1 & Year 2 \\
Effect Typing & Year 2 & Year 4 \\
\hline
\multicolumn{3}{|c|}{Theory}\\
\hline
Meta-theoretic properties & In progress & Year 4 \\
\hline
\multicolumn{3}{|c|}{Applications}\\
\hline
Synthesizing functional program benchmarks & In progress & Year 4 \\
Synthesizing configuration transformations & Year 2 & Year 3 \\
Synthesizing Augeas Lenses & Year 3 & Year 4 \\
\hline
\end{tabular}
\caption{Timeline for investigation and completion of research tasks.}
\label{fig:timeline}
\end{table}
